\section{Theorie}
\label{sec:Theorie}

\subsection{Besetzungsinversion}
Eine Besetzungsinversion bedeutet, dass Zustände mit einer höheren Energien von mehr Teilchen besetzt werden als die Grundzustandsenergie. Dieser Zustand kann nicht im thermischen Gleichgewicht vorkommen, 
sondern nur wenn das System angeregt wird. In dem Fall eines Diodenlasers geschieht dies durch anlegen eines äußeren Stroms an den Halbleiter. Durch das sogenannte Pumpen werden Elektronen aus dem Grundzustand in einen angeregten Zustand gehoben.
Ein wichtiges Kriterium für eine Besetzungsinversion ist die Anzahl der Energieniveaus. So kann es in einem Zwei-Niveau-System nicht zu einer Inversion kommen, da der angeregte Zustand nahezu instantan durch stimulierte Emission, welche durch einstrahlende Photonen
ausgelöst wird, wieder in den Grundzustand zurückfällt. Folglich kann das Phänomen erst ab einem Drei-Niveau-System auftreten - hierbei ist der obererste Zustand sehr kurzlebig, aber anstatt wie im Zwei-Niveau-System auf den Grundzustand zurückzufallen, findet ein Übergang in das zweite angeregte
Niveau statt. Mehr-Niveau-Systeme verhalten sich analog zu jenen mit drei Energieniveaus.

\subsection{p- und n-Dotierung}
Die p- und n-Dotierung sind ein wichtiger Bestandteil der Halbleiter-Physik. Hierbei geht es darum ein Halbleiter, meist aus 4-wertigen Materialien bestehend (zum Beispiel Silizium oder Gallium), durch ein gezieltes Dotieren mit Fremdatomen so anzupassen, dass dieser die benötigten Eigenschaften erhält.
Bei der n-Dotierung werden Atome aus der fünften Hauptgruppe an Gitterplätze des Halbleiter platziert. Da ein solches Atom ein Valenzelektron mehr hat als das ursprüngliche Material, steht dem Material pro Dot ein freies Elektron mehr zur Verfügung sofern eine Spannung angelegt wird.
Die p-Dotierung erfolgt durch das Einfügen eines dreiwertigen Atoms. Das neu platzierte Atom hat ein Valenzelektron weniger, wird also ein weiteres Elektron aufnehmen. Das Loch im Valenzband erzeugt ein weiteres Akzeptorniveau in der Nähe des Valenzbandes. Das dadurch entstehende Loch bewegen sich entgegengesetzt zu den Elektronen.

\subsection{Diodenlaser}
Der Diodenlaser besteht aus einem Halbleiter-Chip, der in \autoref{fig:LaserChip} 
dargestellt ist.
\begin{figure}
    \centering
    \includegraphics[width=0.8\textwidth]{LaserChip.png}
    \caption{Schematischer Aufbau eines Diodenlaser-Chips \cite{ap60}.}
    \label{fig:LaserChip}
\end{figure}
Dieser besteht aus einer p-dotierten und einer n-dotierten Schicht. Der Übergang 
dazwischen, der pn-Übergang, ist das aktive Medium. An einem der langen Enden befindet sich ein 
undurchlässiger Reflektor, an der gegenüberliegenden Seite ein halbdurchlässiger Reflektor. Dies wird über 
polieren der Seiten realisiert. Auf der halbdurchlässigen Seite kann das verstärkte Laserlicht austreten.
\\
\\
Der Chip ist Teil der Littrow-Konfiguration, die in \autoref{fig:Littrow} abgebildet ist. 
\begin{figure}
    \centering
    \includegraphics[width=0.8\textwidth]{LittrowSetup.png}
    \caption{Schematischer Aufbau des Laser-Systems (Littrow-Konfiguration) \cite{ap60}.}
    \label{fig:Littrow}
\end{figure}
Außerhalb des Chips befinden sich eine Kollimator-Linse, die den Laserstrahl zu einer 
parallelen Welle macht, und ein Beugungsgitter. Wenn die Welle auf das Gitter trifft, wird 
das nullte Maximum aus dem Laser hinaus gelenkt, während das Maximum erster Ordnung zurück in den 
Laser reflektiert wird. Dessen Wellenlänge wird durch 
\begin{equation*}
    \lambda = 2dsin(\theta)
\end{equation*}
festgelegt, wobei d die Gitterkonstante und $\theta$ der Winkel des Gitters sind. Es bildet sich 
eine weitere stehende Welle zwischen dem Gitter und dem undurchlässigen Reflektor am anderen Ende des 
Chips aus. Dies nennt man auch den externen Resonator.

\subsection{Nettoleistung der verschiedenen Bauteile}
In \autoref{fig:Gain} wird die Nettoleistung des internen und externen Resonators sowie des Gitters in Abhängigkeit von der Wellenlänge dargestellt. 
Außerdem wird die Nettoleistung ohne zusätzliche Bauteile abgebildet. Die einzelnen Kurven sind relativ zueinander dargestellt.
\begin{figure}
    \centering
    \includegraphics[width=0.8\textwidth]{NetGain.png}
    \caption{Beiträge verschiedener Bauteile zur Nettoleistung \cite{ap60}.}
    \label{fig:Gain}
\end{figure}
Die im aktiven Medium emittierte Strahlung hat ein breites Spektrum mit einem Maximum bei der Wellenlänge 
$\lambda$, die über die Bandlücke bestimmt wird. Die Energie der Bandlücke entspricht 
\begin{equation*}
    E = \frac{\symup{hc}}{\lambda} \, \, ,
\end{equation*}
wobei h das Plancksche Wirkungsquantum ist. Das breite Spektrum ergibt sich dadurch, dass die Elektronen sich in Bändern
befinden und nicht auf diskreten Niveaus. So höher ein Elektron im Leitungsband liegt, desto höher ist dessen Energie.
Das Gitter hat sein Maximum in der Nullten Ordnung, welches als einziges aus dem Laser emittiert wird. Alle weiteren Ordnungen werden
in andere Richtungen gestreut und haben so keinen Einfluss auf die Nettoleistung des Laserstrahls. 
Aus diesem Grund ergibt sich ein Peak bei einer einzigen Wellenlänge.
Der innere Resonator verstärkt nur die Wellenlängen, die die Bedingung der stehenden Welle 
\begin{equation*}
    L = \frac{\lambda}{2}N
\end{equation*}
erfüllen. Dabei ist L die Länge des Resonators. Diese Bedingung gilt auch für den externen Resonator, da dieser aber deutlich größer ist, 
werden wesentlich mehr Wellenlängen in kürzerem Abstand verstärkt. 
\\
\\
In \autoref{fig:Temp} wird die Wellenlänge in Abhängigkeit von der Temperatur dargestellt. 
Die schwarzen Balken beschreiben dabei die Position des Peaks $\lambda$ wie in \autoref{fig:Gain} 
zu sehen. 
\begin{figure}
    \centering
    \includegraphics[width=0.8\textwidth]{ModeTemp.png}
    \caption{Einfluss der Temperatur auf das Modenspektrum und die Wellenlänge \cite{ap60}.}
    \label{fig:Temp}
\end{figure}
Wenn die Temperatur erhöht wird, hat dies einen Einfluss auf die Bandlücke sowie auf den Brechungsindex des 
Mediums. Dies beeinflusst das Modenspektrum des inneren Resonators. Da die Energiebandabstände sich stärker verändern 
als das Modenspektrum, kommt es zu Modensprüngen.
\\
\\
Auch der Pumpstrom hat einen Einfluss auf den Peak der Wellenlänge. Diese Abhängigkeit ist in \autoref{fig:Curr}
zu sehen.
\begin{figure}
    \centering
    \includegraphics[width=0.8\textwidth]{ModeCurr.png}
    \caption{Einfluss des Pump-Stroms auf die Wellenlänge \cite{ap60}.}
    \label{fig:Curr}
\end{figure}
Ein erhöhter Pumpstrom führt nicht nur zu einer erhöhten Temperatur, sondern auch zu einer erhöhten 
Ladungsdichte. Auch dies hat einen Einfluss auf die Wellenlänge und es kommt erneut zu Modensprüngen.
\\
\\
Die Überlagerung der Einflüsse der einzelnen Bauteile befindet sich in \autoref{fig:GainIdeal}. 
Dabei wird die Nettoleistung des inneren Resonators als gestrichelte Linie dargestellt. Als durchgezogene 
Linie wird die Nettoleistung des Gitters und die Nettoleistung des externen Resonators überlagert. 
Das Gitter hat einen Peak bei einer einzigen Wellenlänge wie in \autoref{fig:Gain}. Die Nettoleistung des externen
Resonators sorgt für diskrete Wellenlängen; dessen Einhüllende ist die Nettoleistung des Gitters.
\begin{figure}
    \centering
    \includegraphics[width=0.8\textwidth]{CavityIdeal.png}
    \caption{Überlagerung des Einflusses des inneren Resonators, des Gitter Feedbacks und des äußeren Resonators \cite{ap60}.}
    \label{fig:GainIdeal}
\end{figure}
Dabei stimmt der Peak der Überlagerung genau mit dem Peak des inneren Resonators überein, sodass es keine Modensprünge gibt.
In \autoref{fig:ModeShifts} wird die Abhängigkeit vom Gitterwinkel dargestellt.
\begin{figure}
    \centering
    \includegraphics[width=0.8\textwidth]{ModeShifts.png}
    \caption{Einfluss der Justierung des Gitters auf das Modenspektrum \cite{ap60}.}
    \label{fig:ModeShifts}
\end{figure}
In der Abbildung a oben links befindet sich der gleiche Graph wie in \autoref{fig:GainIdeal}.
Die Moden des internen Resonators wurden mit Int0 und Int1 beschriftet, die Moden des externen Resonators werden mit e-2, e-1, ..., e4
bezeichnet. In Abbildung a befindet sich die Mode e0 genau unterhalb von Int0
Nun wird der Gitterwinkel reduziert, wodurch die Moden des externen Resonators zu einer höheren Frequenz verschoben werden. 
In Abbildung b oben rechts wurde der Gitterwinkel soweit reduziert, dass die Mode Int0 sich genau mittig zwischen den Moden e-1 und e0 befindet. 
Dies führt zu einem Modensprung zu der Mode e-1. Wenn der Gitterwinkel nun weiter reduziert wird, verschieben sich die Moden des externen Resonators weiter zu 
höheren Frequenzen. In Abbildung c findet entsprechend ein Modensprung zu e-2 statt. In Abbildung d unten rechts wurde der Winkel 
so weit reduziert, dass die Mode e3 des externen Resonators genau mittig unter der Mode Int1 des inneren Resonators liegt. Dies führt 
zu einem großen Modensprung zu der Mode e3.

\subsection{Absorptionsspektrum von Rubidium}
Auf der rechten Seite von \autoref{fig:RbSpectrum} ist das zu erwartende Spektrum von Rubidium dargestellt. Dieses wird durch Abfälle der Leistung
der transmittierten Strahlung dargestellt. Die Energieniveaus der beiden Isotope von Rubidium sind links in der Abbildung zu sehen. Die 
Energieniveaus liegen dicht beieinander.
\begin{figure}
    \centering
    \includegraphics[width=0.9\textwidth]{SpectrumIdeal.png}
    \caption{Spektrum von Rubidium \cite{ap60}.}
    \label{fig:RbSpectrum}
\end{figure}