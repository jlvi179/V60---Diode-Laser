\section{Diskussion}
\label{sec:Diskussion}

Allgemein kann man sagen, dass die experimentell bestimmten Ergebnisse mit denen aus der Theorie übereinstimmen.
Der Licht des Lasers zeigt erst ab dem Schwellenstrom das typische Verhalten eines Laserstrahls, welches durch die Kohärenz des Lichtes erfolgt. UNterhalb des Schwellenstroms strahlt der Laser wie eine LED.
Das Transmissionsspektrum passt ebenfalls zu der theoretischen Vorhersage. Es sind die klaren Peaks aus \autoref{fig:RbSpectrum} zu erkennen.
Außerdem kann man an der deutlichen geraden Form des Spektrums erkennen, dass durch die Verwendung der beiden PPhotodioden ein Großteil des Hintergrundes herausgefiltert werden konnte.

Die Ergebnisse dieses Versuches sind also sehr deckungsgleich mit den in \autoref{sec:Theorie} beschriebenen Erwartungen.